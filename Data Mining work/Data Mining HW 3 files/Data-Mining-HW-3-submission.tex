% Options for packages loaded elsewhere
\PassOptionsToPackage{unicode}{hyperref}
\PassOptionsToPackage{hyphens}{url}
%
\documentclass[
]{article}
\usepackage{lmodern}
\usepackage{amsmath}
\usepackage{ifxetex,ifluatex}
\ifnum 0\ifxetex 1\fi\ifluatex 1\fi=0 % if pdftex
  \usepackage[T1]{fontenc}
  \usepackage[utf8]{inputenc}
  \usepackage{textcomp} % provide euro and other symbols
  \usepackage{amssymb}
\else % if luatex or xetex
  \usepackage{unicode-math}
  \defaultfontfeatures{Scale=MatchLowercase}
  \defaultfontfeatures[\rmfamily]{Ligatures=TeX,Scale=1}
\fi
% Use upquote if available, for straight quotes in verbatim environments
\IfFileExists{upquote.sty}{\usepackage{upquote}}{}
\IfFileExists{microtype.sty}{% use microtype if available
  \usepackage[]{microtype}
  \UseMicrotypeSet[protrusion]{basicmath} % disable protrusion for tt fonts
}{}
\makeatletter
\@ifundefined{KOMAClassName}{% if non-KOMA class
  \IfFileExists{parskip.sty}{%
    \usepackage{parskip}
  }{% else
    \setlength{\parindent}{0pt}
    \setlength{\parskip}{6pt plus 2pt minus 1pt}}
}{% if KOMA class
  \KOMAoptions{parskip=half}}
\makeatother
\usepackage{xcolor}
\IfFileExists{xurl.sty}{\usepackage{xurl}}{} % add URL line breaks if available
\IfFileExists{bookmark.sty}{\usepackage{bookmark}}{\usepackage{hyperref}}
\hypersetup{
  pdftitle={Data Mining HW 3},
  pdfauthor={Patrick Chase},
  hidelinks,
  pdfcreator={LaTeX via pandoc}}
\urlstyle{same} % disable monospaced font for URLs
\usepackage[margin=1in]{geometry}
\usepackage{graphicx}
\makeatletter
\def\maxwidth{\ifdim\Gin@nat@width>\linewidth\linewidth\else\Gin@nat@width\fi}
\def\maxheight{\ifdim\Gin@nat@height>\textheight\textheight\else\Gin@nat@height\fi}
\makeatother
% Scale images if necessary, so that they will not overflow the page
% margins by default, and it is still possible to overwrite the defaults
% using explicit options in \includegraphics[width, height, ...]{}
\setkeys{Gin}{width=\maxwidth,height=\maxheight,keepaspectratio}
% Set default figure placement to htbp
\makeatletter
\def\fps@figure{htbp}
\makeatother
\setlength{\emergencystretch}{3em} % prevent overfull lines
\providecommand{\tightlist}{%
  \setlength{\itemsep}{0pt}\setlength{\parskip}{0pt}}
\setcounter{secnumdepth}{-\maxdimen} % remove section numbering
\usepackage{amsmath}
\usepackage{booktabs}
\usepackage{caption}
\usepackage{longtable}
\ifluatex
  \usepackage{selnolig}  % disable illegal ligatures
\fi

\title{Data Mining HW 3}
\author{Patrick Chase}
\date{4/7/2021}

\begin{document}
\maketitle

\hypertarget{section}{%
\section{1.}\label{section}}

That approach would not account for lots of things, such as the
differing styles of policing in cities, local policy, having a baseline
comparison or counterfactual. In short, it may point at some
relationship existing but the specific question of ``increased police
presence causes x to happen'' can't really be based off of that
analysis. Not only that, the question of generalizability is always an
important one to ask. Maybe the impacts of police presence are
heterogenous.

\hypertarget{section-1}{%
\section{2.}\label{section-1}}

Due to a policy decisions to increase police presence around Washington,
D.C. because of higher threats of terrorist activity and not because of
increased crimes, an opportunity for the researchers to conduct a
natural experiment presented itself. The researchers compared crime
rates on days with low terrorist threat levels to crime rates on days
with high terrorist threat levels, controlling for day of the week and
metro ridership. Table 2 presents these two simple models where column 1
regresses total daily crimes on alert levels controling for day of week
and column two adds a measure of ridership to the the model. They found
that on high alert days there were roughly 7 less crimes committed
compared low alert days.

\hypertarget{section-2}{%
\section{3.}\label{section-2}}

They chose to control for Metro ridership because they believed it was
possible that tourists avoided D.C. on days when it was publicized that
the risk level was increased on a given day. Their hypothesis was that
less tourists could lead to less crimes being commited. As a result, the
researchers tried to control for this and compare days with similar
threat levels AND similar ridership.

\hypertarget{section-3}{%
\section{4.}\label{section-3}}

The model is estimating the differing impacts within districts
accounting for district level fixed effects. It seems that a
disproportionate amount of the decrease in criminal activity occurs in
District 1.

\hypertarget{green-building-model}{%
\section{5. Green building model}\label{green-building-model}}

\hypertarget{overview}{%
\subsection{Overview}\label{overview}}

The goal of this model is to estimate the return to investing in green
certification. Specifically, we want to estimate the average change in
rental per square foot given green certification (LEED or EnergyStar).

\hypertarget{data-and-model}{%
\subsection{Data and Model}\label{data-and-model}}

The data we will be using is a collection of 7894 commercial rental
properties. In this data set 685 rentals, or approximately 9\% of the
properties, are green certified. For this analysis we will only consider
if a building has any green certification and not compare between LEED
or EnergyStar.

Lets begin with some linear regressions to get a feel of the relevant
relationships. Models 1 regresses rent per square foot on green
certification. Model 2 regresses rent per square foot on all variables,
excluding CS\_PropertyID, LEED, and EnergyStar. Model 3 uses stepwise
selection on Model 2 to choose the most impactful variables

\captionsetup[table]{labelformat=empty,skip=1pt}
\begin{longtable}{rrr}
\caption*{
\large RMSE\\ 
\small \\ 
} \\ 
\toprule
Model 1 & Model 2 & Model 3 \\ 
\midrule
1602.516 & 296.7834 & 296.8826 \\ 
\bottomrule
\end{longtable}

\hypertarget{results}{%
\subsection{Results}\label{results}}

\hypertarget{conclusion}{%
\subsection{Conclusion}\label{conclusion}}

\hypertarget{california-housing-model}{%
\section{6. California Housing Model}\label{california-housing-model}}

\hypertarget{overview-1}{%
\subsection{Overview}\label{overview-1}}

\hypertarget{data-and-model-1}{%
\subsection{Data and Model}\label{data-and-model-1}}

\hypertarget{results-1}{%
\subsection{Results}\label{results-1}}

\hypertarget{conclusion-1}{%
\subsection{Conclusion}\label{conclusion-1}}

\end{document}
